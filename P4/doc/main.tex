\input{estilo.tex}
\usepackage{textcomp}
\usepackage{listings}
\usepackage{color}

\definecolor{mygreen}{rgb}{0,0.6,0}
\definecolor{mygray}{rgb}{0.5,0.5,0.5}
\definecolor{mymauve}{rgb}{0.58,0,0.82}

\lstset{ %
  backgroundcolor=\color{white},   % choose the background color; you must add \usepackage{color} or \usepackage{xcolor}; should come as last argument
  breakatwhitespace=false,         % sets if automatic breaks should only happen at whitespace
  breaklines=true,                 % sets automatic line breaking
  captionpos=b,                    % sets the caption-position to bottom
  commentstyle=\color{mygreen},    % comment style
  deletekeywords={...},            % if you want to delete keywords from the given language
  escapeinside={\%*}{*)},          % if you want to add LaTeX within your code
  extendedchars=true,              % lets you use non-ASCII characters; for 8-bits encodings only, does not work with UTF-8
  frame=single,	                   % adds a frame around the code
  keepspaces=true,                 % keeps spaces in text, useful for keeping indentation of code (possibly needs columns=flexible)
  keywordstyle=\color{blue},       % keyword style
  language=C,                 % the language of the code
  morekeywords={*,...},            % if you want to add more keywords to the set
  numbers=left,                    % where to put the line-numbers; possible values are (none, left, right)
  numbersep=5pt,                   % how far the line-numbers are from the code
  numberstyle=\tiny\color{mygray}, % the style that is used for the line-numbers
  rulecolor=\color{black},         % if not set, the frame-color may be changed on line-breaks within not-black text (e.g. comments (green here))
  showspaces=false,                % show spaces everywhere adding particular underscores; it overrides 'showstringspaces'
  showstringspaces=false,          % underline spaces within strings only
  showtabs=false,                  % show tabs within strings adding particular underscores
  stringstyle=\color{mymauve},     % string literal style
  tabsize=2,	                   % sets default tabsize to 2 spaces
}

\usepackage[table,xcdraw]{xcolor}


\newpage         


%----------------------------------------------------------------------------------------
%	INDICE
%----------------------------------------------------------------------------------------

\begin{document}
	
\setcounter{page}{0}

\begin{titlepage}
 
 
\newlength{\centeroffset}
\setlength{\centeroffset}{-0.5\oddsidemargin}
\addtolength{\centeroffset}{0.5\evensidemargin}
\thispagestyle{empty}

\noindent\hspace*{\centeroffset}\begin{minipage}{\textwidth}

\centering
\includegraphics[width=0.9\textwidth]{images/logo_ugr.jpg}\\[1.4cm]

\textsc{ \Large CLOUD COMPUTING: SERVICIOS Y APLICACIONES\\[0.2cm]}
\textsc{ MÁSTER EN INGENIERÍA INFORMÁTICA }\\[1cm]
% Upper part of the page
% 
% Title
{\Huge\bfseries Computación	Distribuida	y	Escalable	con	Hadoop	}
\noindent\rule[-1ex]{\textwidth}{3pt}\\[3.5ex]
{\large\bfseries Práctica 4}
\end{minipage}

\vspace{4.5cm}
\noindent\hspace*{\centeroffset}\begin{minipage}{\textwidth}
\centering

\textbf{Autor}\\ {Juan Pablo Porcel Porcel - juanpiporcel@correo.ugr.es}\\[1cm]
\includegraphics[width=0.3\textwidth]{images/etsiit_logo.png}\\[0.1cm]
\textsc{Escuela Técnica Superior de Ingenierías Informática y de Telecomunicación}\\
\textsc{---}\\
Granada, 22 de mayo de 2017
\end{minipage}
%\addtolength{\textwidth}{\centeroffset}
%\vspace{\stretch{2}}
\end{titlepage}




\thispagestyle{empty}

\newpage %inserta un salto de página

\tableofcontents % para generar el índice de contenidos

%\listoffigures

\newpage

%----------------------------------------------------------------------------------------
%	DOCUMENTO
%----------------------------------------------------------------------------------------

\section{Introducción}

El objetivo de esta práctica es realizar programas escalables para mejorar la eficiencia en entornos Big Data. Para ello, se hará uso del entorno que se ha convertido en un estándar de facto como es Hadoop, utilizando HDFS como sistema de ficheros distribuido y Hadoop-MapReduce como mecanismo de ejecución. \\

Para ello el alumno deberá realizar las tareas que se describen a continuación y entregar documentación describiendo con el mayor detalle posible todas las actividades realizadas. \\

Junto con este documento se adjunta el código fuente Java asociado con las diferentes tareas MapReduce. \\


\section{Tareas a realizar}

Utilizando como base el conjunto de datos ECBDL14 situado en la carpeta \textit{/tmp/BDCC/datasets/ECBDL14/ECBDL14\_10tst.data} se pide obtener los siguientes datos estadísticos descriptivos.

\subsection{Mínimo}

Calcular el valor mínimo de la variable (columna) 5. El código asociado a esta tarea se encuentra en el directoro \textit{/tareas/min/}. \\

Salida tras la ejecución del MapReduce:

\begin{lstlisting}
min -11
\end{lstlisting}

\subsection{Máximo}

Calcular el valor máximo de la variable (columna) 5. El código asociado a esta tarea se encuentra en el directoro \textit{/tareas/max/}. \\

Salida tras la ejecución del MapReduce:

\begin{lstlisting}
max 9.0
\end{lstlisting}


\subsection{Máximo y mínimo de una variable}

Calcular al mismo tiempo los valores máximo y mínimo de la variable (columna) 5. El código asociado a esta tarea se encuentra en el directoro \textit{/tareas/maxmin/}. \\

Salida tras la ejecución del MapReduce:

\begin{lstlisting}
min	-11.0
max	9.0
\end{lstlisting}


\subsection{Máximo y mínimo de todas las variables}

Calcular al mismo tiempo los valores máximo y mínimo de todas las variables. El código asociado a esta tarea se encuentra en el directoro \textit{/tareas/maxmin2/}. \\

Salida tras la ejecución del MapReduce:

\begin{lstlisting}
min 1	0.0
max 1	0.154
min 2	-12.0
max 2	10.0
min 3	-11.0
max 3	8.0
min 4	-12.0
max 4	9.0
min 5	-11.0
max 5	9.0
min 6	-13.0
max 6	9.0
min 7	-12.0
max 7	9.0
min 8	-12.0
max 8	7.0
min 9	-13.0
max 9	10.0
min 0	0.094
max 0	0.768
\end{lstlisting}


\subsection{Media}

Calcular la media de la variable 5. El código asociado a esta tarea se encuentra en el directoro \textit{/tareas/average/}. \\

Salida tras la ejecución del MapReduce:

\begin{lstlisting}
average	-1.282261707288373
\end{lstlisting}


\subsection{Media de todas las variables}

Calcular la media de todas las variables. El código asociado a esta tarea se encuentra en el directoro \textit{/tareas/average2/}. \\

Salida tras la ejecución del MapReduce:

\begin{lstlisting}
average 1	0.05212776590932363
average 2	-2.188240380935686
average 3	-1.408876789776933
average 4	-1.7528724942777865
average 5	-1.282261707288373
average 6	-2.293434905140485
average 7	-1.5875789403216172
average 8	-1.7390052924221087
average 9	-1.6989002790625127
average 0	0.25496195991794673
\end{lstlisting}


\subsection{Baleanceo de clase}

Comprobar si el conjunto de datos ECBDL es balanceado o no balanceado, es decir, que el ratio entre las clases sea menor o mayor que 1.5 respectivamente. El código asociado a esta tarea se encuentra en el directoro \textit{/tareas/balanced/}. \\

Salida tras la ejecución del MapReduce:

\begin{lstlisting}
balanced	58.582560602010815
\end{lstlisting}

Se trata de un conjunto de datos con clases no balanceadas ya que el valor que resulta tras la ejecución es mucho mayor de 1.5. \\


\subsection{Correlación}

Calcular el coeficiente de correlación entre todas las parejas de variables. El código asociado a esta tarea se encuentra en el directoro \textit{/tareas/correlation/}. \\

Salida tras la ejecución del MapReduce:

\begin{lstlisting}
correlationcorrelation 0 with 1	-0.13589916869250343
correlation 1 with 2	-0.003036453929943773
correlation 2 with 3	-0.01726247486762999
correlation 3 with 4	0.015754379166559307
correlation 4 with 5	0.07125079800784533
correlation 5 with 6	0.03200113594875155
correlation 6 with 7	0.11488805268078417
correlation 7 with 8	-0.3292179447994215
correlation 8 with 9	0.1084960047958963
correlation 0 with 2	0.09143593112372793
correlation 1 with 3	0.009438349445267866
correlation 2 with 4	0.018191261366109063
correlation 3 with 5	0.016128930425374947
correlation 4 with 6	0.018264386288745375
correlation 5 with 7	0.03297998768398484
correlation 6 with 8	0.07783431570283235
correlation 7 with 9	0.08936167755929571
correlation 0 with 3	0.07005931837948814
correlation 1 with 4	0.058856701856431555
correlation 2 with 5	0.024182999250758484
correlation 3 with 6	0.025952003813569456
correlation 4 with 7	0.01984291578033614
correlation 5 with 8	0.015183324110128226
correlation 6 with 9	0.1071360896407867
correlation 0 with 4	0.04742917824074476
correlation 1 with 5	0.014659977635554567
correlation 2 with 6	0.041153841377462724
correlation 3 with 7	0.01879122854336587
correlation 4 with 8	0.01224584385595619
correlation 5 with 9	0.023068393377281653
correlation 0 with 5	0.12916572716966565
correlation 1 with 6	-0.03183255331046186
correlation 2 with 7	0.03814283037771738
correlation 3 with 8	0.016130402799924542
correlation 4 with 9	0.014041854998880898
correlation 0 with 6	0.19252517591425605
correlation 1 with 7	-1.75036573244801E-5
correlation 2 with 8	0.025077384911599235
correlation 3 with 9	0.01817123896585364
correlation 0 with 7	0.17921266564263214
correlation 1 with 8	0.015894103489103417
correlation 2 with 9	0.027549270387458427
correlation 0 with 8	0.06624560108648203
correlation 1 with 9	-0.016730623449595243
correlation 0 with 9	0.13827089964433972
\end{lstlisting}

\subsection{Otras tareas}

Además de las tareas descritas anteriormente se han realizado las siguientes tareas:

\begin{enumerate}
	\item Parametrizar la columna sobre la que se quiere calcular el estadístico. Esta tarea se ha realizado sobre el código del cálculo del mínimo. El código fuente se encuentra disponible en el directorio \textit{/tareas/min\_parametrizado/}.
	\item Combinar el cálculo de todos los estadísticos en una única función. Se han combinado las operaciones de cálculo del mínimo, máximo y media. El código fuente se encuentra disponible en el directorio \textit{/tareas/min\_max\_average/}. 
	\item Repetir el proceso sobre un conjunto de mayor volumen y comprobar si hay diferencias de tiempo. Se ha calculado el máximo de una variable para los datasets de ECBDL14 y HiggsImg10 y los resultados son los siguientes:
\begin{lstlisting}
ECBDL14		19s
HiggsImg10  16s
\end{lstlisting}
\end{enumerate}

\end{document}
